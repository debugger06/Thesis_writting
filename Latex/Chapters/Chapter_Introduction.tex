% Chapter Template

\chapter{Introduction} % Main chapter title

\label{ChapterX} % Change X to a consecutive number; for referencing this chapter elsewhere, use \ref{ChapterX}

%----------------------------------------------------------------------------------------
%	SECTION 1
%----------------------------------------------------------------------------------------

Let us imagine a science fiction. Astronauts are coming back from World 2.0 after successfully completing their mission. All the crew were so happy that they did not notice a large Asteroid is coming directly towards their space shuttle. When they have detected the astroid, it is already too late to avoid it's axis. So the only possibility is to use their advanced weapon system to destroy the Asteroid. But the battery requires considerable amount of time to initiate the weapon. Luckily they have alternative solar cells and they are equally distributed over the spaceship. Astronauts quickly analyze the situation. They found that, they are crossing a bright star at the moment and they have only chance if they rotate the shuttle such that it maximizes the facing area with respect to the light source. Now, Everything goes accordingly and the space shuttle could avoid the collision. A new problem appears when astonauts identify toxic radiations from the nearby star. So they need to control their appearance to the star such that it can minimise the rays directly falling onto the surface. 


This is the problem where a source in $\mathbb{R}^3$ can be controlled in such a way that, the shadow area to a particular direction of projection can be minimized or maximized. Scientifically the source is considered inert and objective is to identify the direction that optimizes the area of the sources's projection on to another object orthogonal to the direction.


Although Projection to a subspace is mathematically expressed as matrix operation, it is not always obvious how to find optimal projection to a subspace for a specific application. This kind of problem appears in for example analysis of Astronomical, linguistic data, also while designing manifold algorithms. This thesis work is related to projection problem amd goal of this thesis was to find a fast optimised projection to special polytope, named birkhoff polytope.

\section{Problem Description}




 
































