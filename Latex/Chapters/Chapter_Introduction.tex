% Chapter Template

\chapter{Introduction} % Main chapter title

\label{ChapterX} % Change X to a consecutive number; for referencing this chapter elsewhere, use \ref{ChapterX}

%----------------------------------------------------------------------------------------
%	SECTION 1
%----------------------------------------------------------------------------------------

Let us imagine a science fiction. Astronauts are coming back from World 2.0 after successfully completing their mission. All the crew were so happy that they did not notice a large Asteroid is coming directly towards their space shuttle. When they have detected the astroid, it is already too late to avid it's axis. So the only possibility is to use their advanced weapon system to destroy the Asteroid. But the battery requires considerable amount of time to initiate the weapon. Luckily they have alternative solar cells and they are equally distributed over the spaceship. Astronauts quickly analyze the situation. They found that, they are crossing a bright star at the moment and they have only chance if they rotate the shuttle such that it maximizes the facing area with respect to the light source.































Projecting to Birkhoff Polytope is a type of 


