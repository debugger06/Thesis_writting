% Chapter Template

\chapter{The Birkhoff Polytope} % Main chapter title

\label{ChapterX} % Change X to a consecutive number; for referencing this chapter elsewhere, use \ref{ChapterX}



%----------------------------------------------------------------------------------------
%	SECTION 1
%----------------------------------------------------------------------------------------

\section{What is Polytopes}

In elementary geometry, a polytope is a geometric object with flat sides, and may exist in any general number of dimensions n as an n-dimensional polytope or n-polytope. For example a two-dimensional polygon is a 2-polytope and a three-dimensional polyhedron is a 3-polytope

Mathematically, A polytope $P\subseteq \mathbb{R}^d$ is the convex hull $P=conv(v_1,...,v_k)$ of a finite set of points $v_1,...,v_k \in \mathbb{R}^d$ . Dually any polytope can be written as the bounded intersection of a finite number of affine half-spaces in the form $P = \{ x\mid Ax \leq b \}$

A proper face of $F$ of a polytope $P$ is the intersection of $P$ with and affine hyperplane $H$ such that $P$ is completely contained in one of the closed half spaces defined by $H$. The empty set and the polytope P a face of $P$. Any face $F$ is itself  polytope. The dimension of a polytope $P \subseteq \mathbb{R}^d$ is the dimension of the minimum affine space containin it. It is full dimensional if its dimension is d.

0-dimensional faces of $P$ are called $vertices$, 1-dimensional faces are edges. Proper faces of maximal dimensional are called facets. $P$ is the convex hull of its vertices, and the vertices of any face are subset of the vertices of $P$. Thus, a polytope has only a finite number of faces. Let $f_i$ be the number of $i$-dimensional faces of $P,0 \leq i \leq dim P-1$. The $f$-vector of a d-dimensional polytope P is the non-negative integral vector $f(P) = (f_0,...,f_(d-1)$.

The face lettice or combinatorial type $\mathcal{L}(P)$ of a polytope P is the partially ordered set of all faces of P (including the empty face and P itself). This defines Eulerian lattice. The following figure shows this lattice:

\textbf{Here will be picture of a eulerian lattice}

It contains all combinatorial information of the polytope. Two polytopes P,$P'$ are combinatorially isomorphic or have the same combinatorial type if their face lattices are isomorphic as posets.

An r-dimensional simplex (or r-simplex) is the convex hull of r+1 affinely independent points in $\mathbb{R}^d$. A polytope is called simplicial if all facets are simplices. It is simple if the dual is simplicial. Equally, a d-dimensilanl polytope P is simple if each vertex is incident to precisely d edges. The d-dimensional 0/1-cube $C^d$ is the convexhull of all d-dimensinal 0/1-vectors. This is a simple d-polytope with $2^d$ vertices and 2d facets. More generally we denote by a d-cube any d-dimensional polytope that is combinatorially isomorphic to the 0/1-cube (it need not be full dimensional).

Let $P_1 \subset \mathbb{R}^{d_1}$ and $P_2 \subset \mathbb{R}^{d_2}$ be two (geometrically realized) polytopes with vertex sets $V(P_1)=\{ v_1,...,v_k$ and $V(P_2)=\{ w_1,...,w_l \}$. With $0^{(d)}$ we denote he d-dimensional zero vector.

The (geometric) product of $P_1$ and $P_2$ is the polytope

\begin{equation}
P_1 \times P_2 = conv( (v_i,w_i)\in \mathbb{R}^{d_1+d_2}\mid 1\leq i \leq k, i \leq j \leq l)
\label{eqn:Einstein}
\end{equation}

This is the same as the set of all points $(v,w)$ for $v \in P_1$ and $w \in P_2$. The (geometric) join of $P_1$ and $P_2$ is the polytope

\begin{equation}
P_1 \star P_2 := conv(P_1 \times \{ 0^{d_2} \} \times \{ 0 \} \cup \{ 0^{d_1} \} \times P_2 \times \{ 1 \} ) \subseteq \mathbb{R}^{d_1+d_2+1}
\label{eqn:Einstein}
\end{equation}

More generally we say that a polytope P is a product or join of two polytopes $P_1$ and $P_2$, if P is combinatorially isomorphic to the geometric prouct or geometric join of some realisations of the face lattices $P_1$, or $P_2$.

If $F$ is face of a polytope $P = \{ x \mid Ax \leqb \} \subseteq \mathbb{R}^d$ and $ \langle c,x \rangle \leq d $ a linear funtional defining F, then the $wedge wedge_F(P)$ of P over F is defined to be the polytope
 
 \begin{equation}
wedge_F(P) = \{ (x,x_0) \in \mathbb{R}^{d+1} \mid Ax\leq b, 0\leq x_0 \leq d- \langle c,x \rangle\}
\label{eqn:Einstein}
\end{equation}

Again, we say more generally that P is wedge of a polytope  Q over some face F of Q if P is combinatorially equivalently to $wedge_F(Q)$

%-----------------------------------
%	SUBSECTION 1
%-----------------------------------
\subsection{Subsection 1}

Nunc posuere quam at lectus tristique eu ultrices augue venenatis. Vestibulum ante ipsum primis in faucibus orci luctus et ultrices posuere cubilia Curae; Aliquam erat volutpat. Vivamus sodales tortor eget quam adipiscing in vulputate ante ullamcorper. Sed eros ante, lacinia et sollicitudin et, aliquam sit amet augue. In hac habitasse platea dictumst.



%----------------------------------------------------------------------------------------
%	SECTION 2
%----------------------------------------------------------------------------------------

\section{What is Birkhoff Polytope}

In this Chapter I am going to describe about Birkhoff polytope $B_n$ which is sometimes considered to be one of the most important polytopes in many sphere. Birkhoff polytope is also called assignment polytope, the polytope of doubly stochastic matrices, or the perfect matching polytope of complete bipartite graph $K_(n,n)$ ,transportation polytope. It surprisingly appears in various branches of mathematics from geometry to enumerative combinatorics to optimisation theory to Statistics.

A Birkhoff polytope $P_n$ is polytope defined by the following equations and equalities:
\begin{equation}
E = mc^{2}
\label{eqn:Einstein}
\end{equation}	


%-----------------------------------
%	SUBSECTION 1
%-----------------------------------
\subsection{Subsection 1}

Nunc posuere quam at lectus tristique eu ultrices augue venenatis. Vestibulum ante ipsum primis in faucibus orci luctus et ultrices posuere cubilia Curae; Aliquam erat volutpat. Vivamus sodales tortor eget quam adipiscing in vulputate ante ullamcorper. Sed eros ante, lacinia et sollicitudin et, aliquam sit amet augue. In hac habitasse platea dictumst.


%-----------------------------------
%	SUBSECTION 2
%-----------------------------------

\subsection{Subsection 2}
Morbi rutrum odio eget arcu adipiscing sodales. Aenean et purus a est pulvinar pellentesque. Cras in elit neque, quis varius elit. Phasellus fringilla, nibh eu tempus venenatis, dolor elit posuere quam, quis adipiscing urna leo nec orci. Sed nec nulla auctor odio aliquet consequat. Ut nec nulla in ante ullamcorper aliquam at sed dolor. Phasellus fermentum magna in augue gravida cursus. Cras sed pretium lorem. Pellentesque eget ornare odio. Proin accumsan, massa viverra cursus pharetra, ipsum nisi lobortis velit, a malesuada dolor lorem eu neque.

%----------------------------------------------------------------------------------------
%	SECTION 3
%----------------------------------------------------------------------------------------

\section{Main Section 2}

Sed ullamcorper quam eu nisl interdum at interdum enim egestas. Aliquam placerat justo sed lectus lobortis ut porta nisl porttitor. Vestibulum mi dolor, lacinia molestie gravida at, tempus vitae ligula. Donec eget quam sapien, in viverra eros. Donec pellentesque justo a massa fringilla non vestibulum metus vestibulum. Vestibulum in orci quis felis tempor lacinia. Vivamus ornare ultrices facilisis. Ut hendrerit volutpat vulputate. Morbi condimentum venenatis augue, id porta ipsum vulputate in. Curabitur luctus tempus justo. Vestibulum risus lectus, adipiscing nec condimentum quis, condimentum nec nisl. Aliquam dictum sagittis velit sed iaculis. Morbi tristique augue sit amet nulla pulvinar id facilisis ligula mollis. Nam elit libero, tincidunt ut aliquam at, molestie in quam. Aenean rhoncus vehicula hendrerit.