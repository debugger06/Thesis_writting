% Chapter Template

\chapter{CVXOPT} % Main chapter title

\label{ChapterX} % Change X to a consecutive number; for referencing this chapter elsewhere, use \ref{ChapterX}

%----------------------------------------------------------------------------------------
%	SECTION 1
%----------------------------------------------------------------------------------------
In this chapter we will introduce you with CVXOPT library. This library is has implemented a lot of important methods which can be used to solve different optimization problems. In the first a general description about this library has been given, then short a description about its modules. At the end algorithm behind the CVXOPT quadratic cone solver has been described.
  
\section{A First View into CVXOPT Library}

CVXOPT is an opensource  library implemented using Python and C for convex optimization. This library can be installed as python package and can be used with interactive Python interpreter, integrate into software as Python extension module, or on the command line by executing python script. The goal of CVXOPT is to make software development straightforward where convex optimization is required by providing package using the power of python as Python's extensive standard library.


\section{CVXOPT Module Structure}
CVXOPT extends the default python matrix objects into \pyobject{matrix} for dense matrices and \pyobject{spmatrix} for sparse matrices. According to the CVXOPT manual, CVXOPT is organised into following different module:

\subsubsection*{\pyobject{cvxopt.blas}}
This is the interface to most of the double-precision real and complex  Basic Linear Algebra Subprograms. Operations performed using BLAS routines can be implemented as a form arithmatic operation which helps to achieve great simplicity. The main two advantage of BLAS interface over other python blas packages is that:
\begin{itemize}
	\item some functions are not just implementation of basic matrix arithmatic. For example, BLAS includes functions that efficiently exploit symmetry or triangular matrix structure.
	\item BLAS module maintains performance difference with other libraries which is significant for large matrices.
\end{itemize}


\subsubsection*{\pyobject{cvxopt.lapack}}
Interface to dense double-precission real and complex linear equation solvers and eigenvalue routines. This module includes methods for solving dense sets of linear equations, for the corresponding matrix factorizations, for sloving least-squares and least-norm problems, for QR factorization, for symmetric eigenvalue problems, singular value decomposition and Schur factorization.

\subsubsection*{\pyobject{cvxopt.fftw}}
Interface to the discrete transform routines from FFTW and contains routines for discrete Fourier, cosine, and sine transforms.

\subsubsection*{\pyobject{cvxopt.amd}}
Interface to the approximate minimum degree ordering routine from AMD.

\subsubsection*{\pyobject{cvxopt.umfpack}}
This module includes four functions for solving sparse non-symmetric sets of linear equations.
\subsubsection*{\pyobject{cvxopt.cholmod}}
It is an interface to the Cholesky factorization routines of the CHOLMOD package. It includes functions for Cholesky factorization of sparse positive definite matrices.

\subsubsection*{\pyobject{cvxopt.solvers}}
Convex optimization routines and optional interfaces to solvers from GLPK, MOSEK, and DSDP5 

\subsubsection*{\pyobject{cvxopt.modeling}}
Routines for specifying and solving linear programs and convex optimization problems with piecewise-linear cost and constraint functions

\subsubsection*{\pyobject{cvxopt.printing}}
Contains functions and parameters that control how matrices are formatted.

\section{CVXOPT Cone Programming Interface}
CVXOPT considers convex optimization problems as the following form:
\begin{equation*}
\begin{aligned}
& \text{minimize}
& & \frac{1}{2}x^{T}Px+q^{T}x \\
& \text{subject to} & &  Gx \leqslant h\\
& & &  Ax = b
\end{aligned}
\end{equation*}

The linear inequality is a generalized inequality with respect to proper convex cone which may include componentwise vector inequalities, second-order cone inequalities, and linear matrix inequalities. The main solvers are \pyobject{conelp} and\pyobject{coneqp} which are used to optimize linear and quadratic cost functions and problems are required to be strictly primal and dual feasible. As the context of the thesis is Quadratic programming, quadratic programming method of \pyobject{coneqp} interface is focused here.

\subsection{Quadratic Cone Programs}
CVXOPT Quadratic cone program routine solves a pair of primal and dual quadratic cone programs:

\begin{equation}
\begin{aligned}
& \text{minimize}
& & \frac{1}{2}x^{T}Px+q^{T}x \\
& \text{subject to} & &  Gx + s = h\\
& & &  Ax = b\\
& & &  s\geqslant 0\\
\end{aligned}
\label{eqn:CVXOPT_qp_1}
\end{equation}
with P positive semidefinite. And corresponding dual problem is
\begin{equation}
	\begin{aligned}
		& \text{maximize}
& & -(\frac{1}{2})(q+G^Tz+A^Ty)^{T}P^{\dagger}(q+G^Tz+A^Ty)-h^Tz-b^Ty\\
& \text{subject to} & &  q+G^Tz+A^Ty \in \text{ range}(P)\\
& & &  s\geqslant 0\\
	\end{aligned}
	\label{eqn:CVXOPT_qp_2}
\end{equation}

x is the primal variable, s is the slack variable. $y$ and $z$ are dual variables. The inequalities are interpreted as $s\in C$,$z\in C$, were $C$ is a cone defined as a cartesian product of a nonnegative orthant, a number of second-order cones, and a number f positive semidefinite cones:
\begin{equation*}
	\begin{aligned}
		C = C_0 \times C_1 \times ... \times C_M \times C_{M+1}\times ... \times C_{M+N}
	\end{aligned}
\end{equation*}
with
\begin{equation*}
	\begin{aligned}
		C_0 &= \lbrace u\in \mathbb{R}^l \mid u_k \geqslant 0, \text{ }k=1,...,l \rbrace,\\
		C_{K+1} &= \lbrace (u_0,u_1) \in \mathbb{R}\times \mathbb{R}^{\tau_k-1} \mid u_0 \geqslant \| u_1\|_2 \rbrace , & k=0,...,M-1\\
		C_{K+M+1} &=\lbrace \text{vec}(u) \mid u\in S_{\+}^{t_k}\rbrace, & & k=0,...,N-2.
	\end{aligned}
\end{equation*}

According to CVXOPT, $vec(u)$ denotes a symmetric matrix $u$ stored as a vector in column major order.

The CVXOPT typical \pyobject{coneqp} interface look like the following:

\begin{lstlisting}
cvxopt.solvers.coneqp(P,q[,G,h[,dims[,A,b[,initvals[,kktsolver]]]]])	
\end{lstlisting}

\pyobject{P} is a square dense or sparse real matrix, \pyobject{q} is a single-column dense real matrix. \pyobject{h} and \pyobject{b} are real single-column dense matrices. \pyobject{G} and \pyobject{A} are real dense or sparse matrices. The default value of \pyobject{G}, \pyobject{h},\pyobject{A} and \pyobject{b} are metrices with zero rows, meaning that there are no inequalities or equality constraints. It is possible to provide custom solver to solve linear equation (KKT equations). \pyobject{coneqp} returns a dictionary that which contains the result and accuracy of the solution.

\subsection{CVXOPT quadratic cone solver algorithm}

The algorithm implemented in the \pyobject{coneqp} solver is primal-dual path-following method based on Nesterov-Todd scaling.  \pyobject{coneqp} solver is built using the following assumptions:

\subsubsection*{Rank assumptions}
It is assumed that,
\begin{equation*}
	\begin{aligned}
		\textbf{rank}(A)=p, & & \textbf{rank}(\begin{bmatrix}
			P & A^T & G^T & 
		\end{bmatrix} )=n
	\end{aligned}
\end{equation*}
where $p$ is the row dimension and $n$ is the dimension of $x$. that means the matrix:
\begin{equation*}
	\begin{aligned}
		\begin{bmatrix}
			P & A^T & G^T\\
			A & 0 & 0\\
			G & 0 & -Q
		\end{bmatrix}
	\end{aligned}
\end{equation*} 
is nonsingular for any positive definite $Q$.

If $\textbf{rank}(A)<p$ the  either the equality constraints in the primal problem are inconsistent or some of the equalities are redundant. If $\textbf{rank}(\begin{bmatrix}P & A^T & G^T & \end{bmatrix} )<n$, then either the eqaulity constraints in the dual problem are inconsistent or some are redundant and the number of primal vriables can be reduced.

The CVXOPT solver can not identify which rank assumptions does not hold, it only rise an exception if the rank conditions are not satisfied.

\subsubsection*{Logarithmic berrier function}
CVXOPT uses the following barrier function for $C$:
\begin{equation*}
	\begin{aligned}
		\phi(u)=\sum_{k=1}^{K}\phi_k(u_k), & \phi_k(u)=
		\left\{
                \begin{array}{ll}
                  -\sum_{j=1}^{p} \text{log } u_j & C_k=R_{\+}^p \\
                  -(1/2)\text{log}(u_0^2-u_1^T u_1) & C_k=\mathcal{Q}_p \\
                  -\text{log det \textbf{mat}}(u) & C_k = \mathcal{S}_p
                \end{array}
              \right.
	\end{aligned}
\end{equation*}
where $\phi(tu) = \phi(u) - m\text{log}t$ for $t>0$ where
\begin{equation*}
	\begin{aligned}
		m = m_1+...+m_k, & m_k=
		\left\{
                \begin{array}{ll}
                  p & C_k=R_{\+}^p \\
                  1 & C_k=\mathcal{Q}_p \\
                  p & C_k = \mathcal{S}_p
                \end{array}
              \right.
	\end{aligned}
\end{equation*}
$m$ is refers to the degree of the cone $C$.

\subsubsection*{Central path}

In CVXOPT the central path for \ref{eqn:CVXOPT_qp_1} is defined by a family of points $(s,x,y,z)$ that saistfy
\begin{equation}
	\begin{aligned}
		\begin{bmatrix}
			0\\
			0\\
			s
		\end{bmatrix}
		+
		\begin{bmatrix}
			P & A^T & G^T\\
			A & 0 & 0\\
			G & 0 & -Q
		\end{bmatrix}
		+\begin{bmatrix}
			x\\
			y\\
			z\\
		\end{bmatrix}
		=
		\begin{bmatrix}
			-c\\
			b\\
			h
		\end{bmatrix}
		,
		& & (s,z)\succ 0, & & z=-\mu g(s)
	\end{aligned}
	\label{eqn:cvxopt_central_path_1}
\end{equation}
for some $\mu \> 0$. CVXOPT Primal-dual algorithms are based on an equivalent definition of central where primal and dual variables appear symmetrically. The symmetric parametrization is achieved by writing $z=-\mu g(s)$ as $s \circ z=\mu \textbf{e}$ and $\textbf{e}$ is the vector $\textbf{e}=(\textbf{e}_1,...,\textbf{e}_k)$,
\begin{equation*}
	\begin{aligned}
		\textbf{e}_k = 
		\left\{
                \begin{array}{ll}
                  (1,1,...,1) & C_k=R_{\+}^p \\
                  (1,0,...,0) & C_k=\mathcal{Q}_p \\
                  \textbf{vec}(I_p) & C_k = \mathcal{S}_p
                \end{array}
              \right.
	\end{aligned}
\end{equation*}

Central path described finally as:

\begin{equation}
	\begin{aligned}
		\begin{bmatrix}
			0\\
			0\\
			s
		\end{bmatrix}
		+
		\begin{bmatrix}
			P & A^T & G^T\\
			A & 0 & 0\\
			G & 0 & -Q
		\end{bmatrix}
		+\begin{bmatrix}
			x\\
			y\\
			z\\
		\end{bmatrix}
		=
		\begin{bmatrix}
			-c\\
			b\\
			h
		\end{bmatrix}
		,
		& & (s,z)\succ 0, & & s\circ z = \mu \textbf{e}
	\end{aligned}
	\label{eqn:cvxopt_central_path_2}
\end{equation}

\subsubsection*{Nesterov-Todd Scaling}

A primal-dual scaling $W$ is a linear transformation
\begin{equation*}
\begin{aligned}
	\bar{s} = W^-Ts, & & \bar{z}=Wz
\end{aligned}
\end{equation*}
that leaves the cone and the central path invariant, $i.e.$,
\begin{equation*}
\begin{aligned}
s\succ 0 \Leftrightarrow \bar{s}\succ,  & & z\succ \Leftrightarrow \bar{z}\succ 0, & & s\circ z=\mu \textbf{e} \Leftrightarrow \bar{s}\circ \bar{z} = \mu \textbf{e}
\end{aligned}
\end{equation*}

When $W$ is a scaling, central path equations has been written equvalently as the following form
\begin{equation}
	\begin{aligned}
		\begin{bmatrix}
			0\\
			0\\
			s
		\end{bmatrix}
		+
		\begin{bmatrix}
			P & A^T & G^T\\
			A & 0 & 0\\
			G & 0 & -Q
		\end{bmatrix}
		+\begin{bmatrix}
			x\\
			y\\
			z\\
		\end{bmatrix}
		=
		\begin{bmatrix}
			-c\\
			b\\
			h
		\end{bmatrix}
		,
		& & (s,z)\succ 0, & & (Wz)\circ (W^{-T}s) = \mu \textbf{e}
	\end{aligned}
	\label{eqn:cvxopt_central_path_3}
\end{equation}

\subsubsection*{Path-following Algorithm for Cone QPs}

The algorithm implemented in CVXOPT $coneqp$ is based on linearizing the central path equation \ref{eqn:cvxopt_central_path_3}. which is achieved from equation \ref{eqn:cvxopt_central_path_2} after applying a scaling with a matrix $W$.

CVXOPT defines the current iterate by $(\hat{s},\hat{x},\hat{y},\hat{z})$. Initial iterate value is $((\hat{s},\hat{x},\hat{y},\hat{z})=(s_0,x_0,y_0,z_0))$ where $s_0\succ 0,z_0 \succ 0$. Neterov-Todd scalling $W$ is calculated at $\hat{s},\hat{z}$ and the scaled variable $\lambda := W^{-T}\hat{s} = W\hat{z}$

The algorithm follows the following steps.

\begin{enumerate}
    \item\textit{Euclidian residuals,gap,and stopping criteria}.Compute\\
    \begin{equation}
    	\begin{aligned}
    		\begin{bmatrix}
        		r_x\\
        		r_y\\
        		r_z
    		\end{bmatrix}
    		=
    		\begin{bmatrix}
        		0\\
        		0\\
        		\hat s
    		\end{bmatrix}
    		+
    		\begin{bmatrix}
        		P & A^\mathsf{T} & G^\mathsf{T}\\
        		A & 0 & 0\\
        		G & 0 & 0
    		\end{bmatrix}
    		\begin{bmatrix}
        		\hat x\\
        		\hat y\\
        		\hat z
    		\end{bmatrix}
    		\begin{bmatrix}
        		c\\
        		-b\\
        		-h
    		\end{bmatrix}
    	\end{aligned}
    	\label{eqn:cvxopt_algorithm_1}
    \end{equation}

    and\\
    \begin{equation*}
    	\begin{aligned}
    		\hat{\mu} = \frac{\hat{s}^\mathsf{T}\hat{z}}{m}=\frac{\lambda^\mathsf{T}}{m}
    	\end{aligned}
    \end{equation*}

    Terminate if $(s,x,y,z)=(\hat s,\hat x,\hat y,\hat z)$ satiesfies (within some threshold value) the optimality conditions
    \begin{equation*}
    	\begin{aligned}
    		\begin{bmatrix}
    			0\\
    			0\\
    			s\\
    		\end{bmatrix}
    		=
    		\begin{bmatrix}
        		P & A^\mathsf{T} & G^\mathsf{T}\\
        		A & 0 & 0\\
        		G & 0 & 0\\
    		\end{bmatrix}
    		\begin{bmatrix}
    			x\\
    			y\\
    			z\\
    		\end{bmatrix}
    		+
    		\begin{bmatrix}
    			c\\
    			b\\
    			h
    		\end{bmatrix}
    		,
    		& & (s,z) \succeq0, & &  z^\mathsf{T}s=0.
    	\end{aligned}	
    \end{equation*}
    \item \textit{Affine Direction.} Solve the linear equations:\\
    \begin{equation}
    	\begin{aligned}
			\begin{bmatrix}
    			0\\
    			0\\
    			\Delta s_a\\
    		\end{bmatrix}
    		+
    		\begin{bmatrix}
        		P & A^\mathsf{T} & G^\mathsf{T}\\
        		A & 0 & 0\\
        		G & 0 & 0\\
    		\end{bmatrix}
    		\begin{bmatrix}
    			\Delta x_a\\
    			\Delta y_a\\
    			\delta z_a\\
    		\end{bmatrix}
    		=-
    		\begin{bmatrix}
    			r_x\\
    			r_y\\
    			r_z\\
    		\end{bmatrix}
    		\\
    		\lambda\circ(W\Delta {z_a} +W^{-\mathsf{T}}\Delta s_a ) = -\lambda \circ \lambda.\\
    	\end{aligned}
    	\label{eqn:cvxopt_algorithm_2}
    \end{equation}
    
    
    \item \textit{Step size and centering parameter.} Cpmpute\\
    	\begin{equation*}
    		\begin{aligned}
    			\alpha &= \textrm{sup} \lbrace \alpha \in[0,1] \mid (\hat{s},\hat{z})+\alpha(\Delta s_a,\Delta z_a)\succeq 0\rbrace \\
    			&= \textrm{sup} \lbrace \alpha \in[0,1]  \mid (\lambda,\lambda)+\alpha ( W^{-T} \Delta s_a, W\Delta z_a)\succeq 0 \rbrace \\
    			\rho &= \frac{(\hat{s}+\alpha \Delta s_a)^T (\hat{z}+\alpha \Delta z_a)^T} {\hat{s}^T\hat{z}}\\
    				&= 1-\alpha+\alpha^2\frac{(W^{-T} \Delta s_a)(W\Delta z_a)}{\lambda^{T}\lambda} \\
    			\sigma &= \max{\lbrace 0,\min{\lbrace 1,\rho \rbrace} \rbrace }^3\\
    		\end{aligned}
    	\end{equation*}
    
    \item \textit{Combined direction}. Solve the linear equation\\
    	\begin{equation}
    		\begin{aligned}
    			\begin{bmatrix}
    				0\\
    				0\\
    				\Delta s_a\\
    			\end{bmatrix}
    			+
    			\begin{bmatrix}
        			P & A^\mathsf{T} & G^\mathsf{T}\\
        			A & 0 & 0\\
        			G & 0 & 0\\
    			\end{bmatrix}
    			\begin{bmatrix}
    				\Delta x_a\\
    				\Delta y_a\\
    				\delta z_a\\
    			\end{bmatrix}
    			=-(1-\eta)
    			\begin{bmatrix}
    				r_x\\
    				r_y\\
    				r_z\\
    			\end{bmatrix}\\
    			\lambda\circ(W\Delta {z_a} +W^{-\mathsf{T}}\Delta s_a ) = -\lambda \circ \lambda - \gamma(W^\mathsf{-T}\Delta s_a )\circ(W\Delta z_a)+\sigma\hat\mu\textbf{e}\\
    		\end{aligned}
    		\label{eqn:cvxopt_algorithm_3}
    	\end{equation}
    	Common choice for $\eta$ are $\eta=0$ and $\eta=\sigma$. The parameter $\eta$ is 1 or 0, depending on whether or not a Mehrota correction is used. The default value is $\gamma=1$
    \item \textit{Update iterators and scalling matrices},
    	\begin{equation*}
    		\begin{aligned}
    			(\hat s,\hat x,\hat y,\hat z):=(\hat s,\hat x,\hat y,\hat z)+\alpha(\Delta s,\Delta x,\Delta y,\Delta z)
    		\end{aligned}
    	\end{equation*}
    Where
    \begin{equation*}
    	\alpha = \textrm{sup} \left\{ {\alpha} {\in[0,1] } | (\lambda,\lambda)+\frac{\alpha}{0.99} ( W^{-\mathsf{T}} \Delta s_a, W\Delta z_a)\succeq 0\right\}\\
    \end{equation*}
    Compute the scalling matrix $W$ for $\hat s$,$\hat z$ and the second variable $\lambda :=W^\mathsf{-T}\hat s = W\hat z $  
\end{enumerate}

\subsubsection*{Initialization of the Algorithm}
If primal and dual starting points $\hat{x},\hat{s},\hat{y},\hat{z}$ are not specified during the calling the method,  they are initialized as by the algorithm by the following way.

\pyobject{coneqp} algorithm solves the linear equation:

\begin{equation}
	\begin{aligned}
		\begin{bmatrix}
			P & A^T & G^T\\
			A & 0 & 0\\
			G & 0 & -I
		\end{bmatrix}
		\begin{bmatrix}
			x\\
			y\\
			z
		\end{bmatrix}
		=
		\begin{bmatrix}
			-c\\
			b\\
			h
		\end{bmatrix}
	\end{aligned}
	\label{eqn:cvxopt_initialization_1}
\end{equation}
and takes $\hat{x} = x, \hat{y}=y$. The equation \ref{eqn:cvxopt_initialization_1} gives the optimality conditions for the pair of primal and dual problems
\begin{equation*}
\begin{aligned}
& \text{minimize}
& & \frac{1}{2}x^{T}Px+c^{T}x\parallel s\parallel_2^2 \\
& \text{subject to} & &  Gx + s = h\\
& & &  Ax = b\\
\end{aligned}
\label{eqn:CVXOPT_qp_1}
\end{equation*}
and
\begin{equation*}
	\begin{aligned}
		& \text{maximize}
& & -(\frac{1}{2})w^TPw-h^Tz-b^Ty-(1/2)\parallel z \parallel_2^2 \\
& \text{subject to} & &  Pw+G^Tz+A^Ty+c=0
	\end{aligned}
	\label{eqn:CVXOPT_qp_2}
\end{equation*}

The initial value of $\hat{s}$ is computed from the residuals $h-Gx=-z$ as
\begin{equation*}
	\begin{aligned}
		\hat{s} = 
		\left\{
                \begin{array}{ll}
                  -z & \alpha_p <0 \\
                  -z+(1+\alpha_p)\textbf{e} & \text{otherwise}
                \end{array}
              \right.
	\end{aligned}
\end{equation*}
where $\alpha_p=\text{inf}\lbrace \alpha\mid-z+\alpha\textbf{e}\succeq 0 \rbrace$. The initial value of $z$ is:

\begin{equation*}
	\begin{aligned}
		\hat{s} = 
		\left\{
                \begin{array}{ll}
                  z & \alpha_d <0 \\
                  -z+(1+\alpha_d)\textbf{e} & \text{otherwise}
                \end{array}
              \right.
	\end{aligned}
\end{equation*}
where $\alpha_d=\text{inf}\lbrace \alpha\mid z+\alpha\textbf{e}\succeq 0 \rbrace$.































